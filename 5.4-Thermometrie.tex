% Klassifiziert den Dokumenten-Typ
% Doku: http://exp1.fkp.physik.tu-darmstadt.de/tuddesign/
% Farben: http://www.tu-darmstadt.de/media/medien_stabsstelle_km/services/medien_cd/das_bild_der_tu_darmstadt.pdf
%  bigchapter: Chapter haben doppelte Schriftgröße
%  linedtoc: Linien im Inhaltsverzeichnis wie bei Überschriften
%  colorbacktitle: Der Dokumenten-Titel wird mir der Accentfarbe hinterlegt
\documentclass[bigchapter,colorback,accentcolor=tud4b,linedtoc,11pt]{tudreport}

% Input Dokument hat das Encoding UTF-8
\usepackage[utf8]{inputenc}
% Wichtiges Paket für Links und verlinktes Inhaltsverzeichnis
\usepackage{hhline}
\usepackage[ngerman]{hyperref}
% Paket für Fußnoten
\usepackage[stable]{footmisc}
% Paket für amsmath (aligned mathe formeln)
\usepackage{amsmath}
% Paket für Bibliotheks-Verzeichnis, square: Verwende eckige statt runde klammern
% \usepackage[square]{natbib}
% Paket zum Plotten von Datensätzen
\usepackage{pgfplots}
\usepgfplotslibrary{patchplots}


\pgfkeys{%
  /pgfplots/default/.style={%
    /pgf/number format/use comma,
    legend pos=north west,
    width=0.9\linewidth,
    height=0.7\linewidth,
    scale only axis,
    xmin=0,
    ymin=0,
    grid=both,
    tick align=outside,
    tickpos=left,
    minor x tick num=3,
    minor y tick num=4,
    minor grid style={dotted,thin},
    x tick label style={/pgf/number format/.cd,%
      set thousands separator={},
      set decimal separator={,}
    },%
    y tick label style={/pgf/number format/.cd,%
      set thousands separator={},
      set decimal separator={,}
    },%
  }
}

% Anhänge für Original-Messdaten
\usepackage{fancyvrb}

% redefine \VerbatimInput
\RecustomVerbatimCommand{\VerbatimInput}{VerbatimInput}%
{fontsize=\footnotesize,
 %
 frame=lines,  % top and bottom rule only
 framesep=2em, % separation between frame and text
 fontsize=\scriptsize,
 %
 labelposition=topline,
 %
 commandchars=\|\(\), % escape character and argument delimiters for
                      % commands within the verbatim
 commentchar=*        % comment character
}

% Polar Plots
\usetikzlibrary{pgfplots.polar}
% Verwende deutsche Bezeichner für Inhaltsverzeichnis, ... (ngerman = New German: neue Rechtschreibung)
\usepackage{ngerman}
% Deutsche Zahlen (entfernt z.B. das Leerzeichen nach einem Dezimal-Komma)
\usepackage{ziffer} 

\usepackage[verbose]{placeins}

%wegen Grafikverschiebung hinzugefügt
\usepackage{float}

%\usepackage{graphicx}
%\usepackage{caption}
\usepackage{subcaption} %Für subfigures

% PDF-Optionen
\hypersetup{%
  pdftitle={TU Darmstadt \- Physikalisches Praktikum für Fortgeschrittene},
  pdfauthor={Esra Bauer und Sören Link},
  pdfsubject={Versuch 5.4},
  pdfview=FitH,
}
% Nummeriere formeln in Subsections einzeln
% Kleines makro zur assymetrischen Fehlerangabe

% Entspricht-Zeichen
\usepackage{scalerel}

\newcommand\equalhat{%
\let\savearraystretch\arraystretch
\renewcommand\arraystretch{0.3}
\begin{array}{c}
\stretchto{
    \scalerel*[\widthof{=}]{\wedge}
    {\rule{1ex}{3ex}}%
}{0.5ex}\\ 
=%
\end{array}
\let\arraystretch\savearraystretch
}
%BEGINN TITELSEITE

\title{Tieftemperaturmessung an suprafluidem Helium}

\subtitle{Esra Bauer \\Sören Link}

\subsubtitle{Betreuer: M.Sc. Michael Lannert \hfill Versuchsdatum: 1. Juni 2015}

\author{Esra Bauer, Sören Link}

%\settitlepicture{img/title.jpg}

\institution{Physikalisches Praktikum \\für Fortgeschrittene \\ Versuch 5.4}

\date{\today}


%ENDE TITELSEITE

\begin{document}
%ANFANG DOKUMENT

%Titelseite einfügen
\maketitle

%Inhaltsverzeichnis einfügen
\tableofcontents

%ANFANG INHALT

\chapter{Einleitung}

In diesem Versuch geht es um Temperaturmessung im Bereich von etwa 1 bis 5 K. Diese Temperaturen erzeugen wir in einem mit flüssigem Helium gefüllten Badkryostaten. Die Messung erfolgt mittels eines sekundären Thermometers in Form einer paramagnetischen Substanz, die zunächst mittels bekannter Messpunkte kalibriert werden muss. Damit können wir schließlich den sog. $\lambda$-Punkt von Helium bestimmen, unterhalb dessen die Suprafluidität eintritt, d.h. das flüssige Helium verliert seine Viskosität und verändert seine Eigenschaften bezüglich Wärmeleitfähigkeit und Wärmekapazität, was auch im Folgenden betrachtet wird.

\chapter{Grundlagen}

\section{Kühlverfahren zum Erreichen der Suprafluidität}

\section{Thermometrie bei tiefen Temperaturen}

\section{Paramagnetismus}

\section{Eigenschaften von $^3$He und $^4$He}

\chapter{Durchführung}

\section{Aufnahme der Abkühlkurve}

\section{Aufnahme der Aufwärmkurve}

\chapter{Auswertung}

\section{Druckkorrektur}

\section{Überprüfung des Curie- und Curie-Weiss-Gesetzes}

\section{Verlauf der Aufwärmung}

\section{Wärmefluss in das Helium}

\chapter{Fazit}

%ENDE INHALT
\cleardoublepage{}
% Eintrag fürs Inhaltsverzeichnis
\newpage
\begin{thebibliography}{100}
  \bibitem{anleitung} Versuchsanleitung zum Versuch Holographie, heruntergeladen am 09.05.2015 von der Homepage des IAP der TU Darmstadt
  \bibitem{blackbodycoherence} The coherence length of black-body
radiation by Axel Donges, 1998: \url{http://hank.uoregon.edu/teaching-modules/Broadband-Interferometer/BBcoherence.pdf}
  \bibitem{boltzmann} Abteilung theoretische Chemie der Goethe-Universität
    Frankfurt von Prof. J. Wachtveitl: \url{http://www.theochem.uni-frankfurt.de/femtochem/lectures/PC2_Sem/Handouts-PCII-WS-12-13/Zotko_Maxwell-Boltzmann%20Verteilung.pdf}
\end{thebibliography}
\end{document}

%%% Local Variables:
%%% mode: latex
%%% TeX-master: t
%%% End:
